% !TEX program = pdflatex
\documentclass[11pt,a4paper,open=any]{scrbook} % KOMA-Script article class
\usepackage[a4paper,margin=2cm]{geometry}

% Pakiety podstawowe
\usepackage[utf8]{inputenc} % kodowanie znaków
\usepackage[T1]{fontenc}
\usepackage[ngerman,english,polish]{babel} % języki
\usepackage{lmodern} % lepsze fonty
\usepackage{hyperref} % linki
\usepackage{graphicx} % grafika
\usepackage{amsmath, amssymb} % matematyka
\usepackage{listings} % kod źródłowy
\usepackage{xcolor}   % kolory
\usepackage{tcolorbox} % kolorowe ramki
\usepackage{tikz} % rysunki
\usepackage{enumitem} % lepsze listy
\usepackage{minted}


% Definicja środowiska dla przykładów
\tcbset{
        tutorial/.style={
        colback=blue!5!white,
        colframe=blue!75!black,
        fonttitle=\bfseries,
        title=Beispiel,
    }
}

\tcbset{
        red/.style={
        colback=red!5!white,
        colframe=red!75!black,
        fonttitle=\bfseries,
        title=Beispiel,
    }
}

\tcbset{
        blue/.style={
        colback=blue!5!white,
        colframe=blue!75!black,
        fonttitle=\bfseries,
        title=Beispiel,
    }
}

% Ustawienia dla listings (kod źródłowy)
\lstset{
    language=[LaTeX]TeX,
    basicstyle=\ttfamily\small,
    backgroundcolor=\color{gray!10},
    showspaces=false,
    keepspaces=false,
    frame=single,
    breaklines=true,
    captionpos=b,
    numbers=none,
    numberstyle=\tiny,
    keywordstyle=\color{blue},
    commentstyle=\color{gray}
}

% Dokument
\begin{document}

%%%%%%%%%%%%%%%%%%%%%%%%%%%%%%%%%%%%%%%%%%%%%
%
% $Autor: Wojciech Zarzycki $
% $Datum: 2020-09-17 12:27:00Z $
% !TEX root = ../LatexTutorial.tex
% $Version: 1 $
%
%%%%%%%%%%%%%%%%%%%%%%%%%%%%%%%%%%%%%%%%%%%%%

\title{Mein eigenes \LaTeX-Tutorial}
\author{Wojciech Zarzycki}
\date{\today}
\maketitle


\tableofcontents
\newpage

\chapter{Introduction}
%%%%%%%%%%%%%%%%%%%%%%%%%%%%%%%%%%%%%%%%%%%%%
%
% $Autor: Wojciech Zarzycki $
% $Datum: 2020-09-17 12:27:00Z $
% !TEX root = ../LatexTutorial.tex
% $Version: 1 $
%
%%%%%%%%%%%%%%%%%%%%%%%%%%%%%%%%%%%%%%%%%%%%%


Hier entsteht mein persönliches \LaTeX-Tutorial.  
Ich erkläre mir selbst die wichtigsten Grundlagen mit Beispielen.
  

\chapter{Key Shortcuts}
%%%%%%%%%%%%%%%%%%%%%%%%%%%%%%%%%%%%%%%%%%%%%
%
% $Autor: Wojciech Zarzycki $
% $Datum: 2020-09-17 12:27:00Z $
% !TEX root = ../LatexTutorial.tex
% $Version: 1 $
%
%%%%%%%%%%%%%%%%%%%%%%%%%%%%%%%%%%%%%%%%%%%%%

\begin{tcolorbox}[red,title={Hinweis}]
In Latex Workshop for VSC you can use multiple Key Combinations to achieve different actions. Here some of them for Apple Devices.
\end{tcolorbox}

\begin{itemize}
    \item CMD + OPT + J $\rightarrow$ Jump to actual place in pdflatex
    \item CMD + OPT + V $\rightarrow$ Show PDF
    \item CMD + OPT + B $\rightarrow$ Compile File
    \item CMD + left click on PDF $\rightarrow$ Jump to the place in code
\end{itemize}




\chapter{Images}
%%%%%%%%%%%%%%%%%%%%%%%%%%%%%%%%%%%%%%%%%%%%%
%
% $Autor: Wojciech Zarzycki $
% $Datum: 2020-09-17 12:27:00Z $
% !TEX root = ../LatexTutorial.tex
% $Version: 1 $
%
%%%%%%%%%%%%%%%%%%%%%%%%%%%%%%%%%%%%%%%%%%%%%

\begin{tcolorbox}[red,title={Hinweis}]
Środowisko figure to tzw. float - czyli obiekt pływający. Dzięki temu wszystkie grafiki są sformatowane zgodnie z normą, kada figura dostaje swój numer i poźniej jest prezentowana w liście rysunków.
\end{tcolorbox}


\section{Add Image}
\begin{lstlisting}
\begin{figure}[H] % oder [htbp]
    \centering
    \includegraphics[width=0.7\linewidth]{path}
    \caption{title}
    \label{fig:label}
\end{figure}

\end{lstlisting}

\section{Image and Tikz}
\begin{lstlisting}
\begin{figure}[H] % oder [htbp], je nach Platzierungswunsch
  \centering
  \begin{tikzpicture}
    % obrazek jako tlo
    \node[anchor=south west, inner sep=0] (image) at (0,0) {%
      \includegraphics[width=0.7\linewidth]{path}};
        
    \begin{scope}[x={(image.south east)}, y={(image.north west)}]
      \draw[red, line width=4pt, <-, >=angle 60] (0.37,0.755) -- ++(0.2,-0.1);
        
      % lib_deps
      \draw[red, line width=2pt] (0.06,0.52) rectangle ++(0.5, 0.08);
        
      % #include <XXXX.h>
      \draw[red, line width=2pt] (0.06,0.32) rectangle ++(0.35, 0.08);
    \end{scope}
  \end{tikzpicture}
  \caption{Title}
  \label{fig:label}
\end{figure}
\end{lstlisting}

\begin{tcolorbox}[blue,title={Hinweis}]
Argument opcjonalny [htbp] (here, top, bottom, page) pozwala LaTeXowi decydować, gdzie najlepiej umieścić rysunek, aby układ był estetyczny:
    \begin{itemize}
        \setlength\topsep{0em}      
        \setlength\itemsep{0.2em}   % odstęp między punktami
        \setlength\parskip{0pt}     % odstęp między akapitami
        \setlength\parsep{0pt}
        
        \item \texttt{[h]} → ,,here'' - jak najbliżej miejsca w kodzie
        \item \texttt{[t]} → ,,top'' - na górze strony
        \item \texttt{[b]} → ,,bottom'' - na dole strony
        \item \texttt{[p]} → osobna strona tylko z rysunkami
    \end{itemize}
Zwykle używa się \texttt{[htbp]}, a dla wymuszenia \texttt{[H]} (wymaga \texttt{\textbackslash usepackage\{float\}}).

\end{tcolorbox}


\chapter{List Environment}
%%%%%%%%%%%%%%%%%%%%%%%%%%%%%%%%%%%%%%%%%%%%%
%
% $Autor: Wojciech Zarzycki $
% $Datum: 2020-09-17 12:27:00Z $
% !TEX root = ../LatexTutorial.tex
% $Version: 1 $
%
%%%%%%%%%%%%%%%%%%%%%%%%%%%%%%%%%%%%%%%%%%%%%

\section{Change spacing}

\begin{lstlisting}
\begin{itemize}
    \setlength\topsep{0em}      % spacing between first and last element
    \setlength\itemsep{0.2em}   % spacing between lines
    \setlength\parskip{0pt}     % spacing between new paragraph
    \setlength\parsep{0pt}

    \item xxx
    \item yyy
\end{itemize}
\end{lstlisting}







% ================================================
\newpage
\begin{tcolorbox}[tutorial,title={Hinweis}]
Alles, was in diesen Boxen steht, sind kurze Notizen oder Tipps für mich.
\end{tcolorbox}

\section{Textformatierung}
Normale Schrift, \textbf{fett}, \textit{kursiv}, \underline{unterstrichen}.  

\begin{tcolorbox}[tutorial,title={Beispiel}]
\begin{verbatim}
    Normale Schrift, \textbf{fett}, \textit{kursiv}, 
    \underline{unterstrichen}.
\end{verbatim}
\end{tcolorbox}

\section{Listen}
\subsection{Nummerierte Liste}
\begin{enumerate}
  \item Punkt 1
  \item Punkt 2
  \item Punkt 3
\end{enumerate}

\subsection{Stichpunkte}
\begin{itemize}
  \item Erster
  \item Zweiter
\end{itemize}

\section{Mathematik}
Inline: $a^2 + b^2 = c^2$  

Abgesetzt:
\[
  \int_0^1 x^2 \, dx = \frac{1}{3}
\]

\section{Tabellen}
\begin{tabular}{|l|c|r|}
\hline
Links & Mitte & Rechts \\
\hline
A & B & C \\
\hline
\end{tabular}

\section{Abbildungen}
\begin{figure}[h]
    \centering
    \includegraphics[width=0.5\textwidth]{example-image}
    \caption{Beispielbild}
\end{figure}

\section{TikZ Beispiel}
\begin{tikzpicture}
  \draw[->] (0,0) -- (2,0) node[right]{x};
  \draw[->] (0,0) -- (0,2) node[above]{y};
  \draw (0,0) circle(1cm);
\end{tikzpicture}

\section{Code-Beispiel}
\begin{lstlisting}[language=C, caption={Hello World in C}]
#include <stdio.h>
int main() {
    printf("Hello, World!\n");
    return 0;
}
\end{lstlisting}

\end{document}
